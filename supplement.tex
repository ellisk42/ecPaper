%%%%%%%% ICML 2018 EXAMPLE LATEX SUBMISSION FILE %%%%%%%%%%%%%%%%%

\documentclass{article}

% Recommended, but optional, packages for figures and better typesetting:
\usepackage{microtype}
\usepackage{graphicx}
\usepackage{subfigure}
\usepackage{booktabs} % for professional tables

% hyperref makes hyperlinks in the resulting PDF.
% If your build breaks (sometimes temporarily if a hyperlink spans a page)
% please comment out the following usepackage line and replace
% \usepackage{icml2018} with \usepackage[nohyperref]{icml2018} above.
\usepackage{hyperref}

% Attempt to make hyperref and algorithmic work together better:
\newcommand{\theHalgorithm}{\arabic{algorithm}}

\newcommand{\system}{\textsc{HelmholtzHacker}~}

% Use the following line for the initial blind version submitted for review:
\usepackage{icml2018}


\usepackage{hyperref}       % hyperlinks
\usepackage{url}            % simple URL typesetting
\usepackage{booktabs}       % professional-quality tables
\usepackage{amsfonts}       % blackboard math symbols
\usepackage{nicefrac}       % compact symbols for 1/2, etc.
\usepackage{microtype}      % microtypography

\usepackage{listings}
\usepackage{amsthm}
% use Times
\usepackage{times}
% For figures
\usepackage{graphicx} % more modern
%\usepackage{epsfig} % less modern
\usepackage{subfig} 
\usepackage{fancyvrb}


\usepackage{caption}
\usepackage{subcaption}

\fvset{fontsize=\footnotesize}

\usepackage{amssymb}
\usepackage{listings}
\usepackage{wrapfig}
\usepackage{tabularx}


\usepackage{verbatim}
 \usepackage{booktabs}
 % For algorithms
\usepackage{algorithm}
\usepackage{algorithmic}
\usepackage{tikz}
\usetikzlibrary{fit,bayesnet}
%\usetikzlibrary{arrows.meta}
%\usetikzlibrary{positioning}
%\usetikzlibrary{decorations.text}
%\usetikzlibrary{decorations.pathmorphing}
\usepackage{dsfont}
\usepackage{amsmath}
\usepackage{hyperref}
\DeclareMathOperator*{\argmin}{arg\,min} % thin space, limits underneath in displays
\DeclareMathOperator*{\argmax}{arg\,max} % thin space, limits underneath in displays
\DeclareMathOperator{\argmin}{argmin} % no space, limits underneath in displays



% Packages hyperref and algorithmic misbehave sometimes.  We can fix
% this with the following command.

\newcommand{\Expect}{\mathds{E}} %{{\rm I\kern-.3em E}}
\newcommand{\indicator}{\mathds{1}} %{{\rm I\kern-.3em E}}
\newcommand{\expect}{\mathds{E}} %{{\rm I\kern-.3em E}}
\newcommand{\probability}{\mathds{P}} %{{\rm I\kern-.3em P}}



% If accepted, instead use the following line for the camera-ready submission:
%\usepackage[accepted]{icml2018}

% The \icmltitle you define below is probably too long as a header.
% Therefore, a short form for the running title is supplied here:
\icmltitlerunning{Inducing Domain Specific Languages for Bayesian Program Learning}

\begin{document}

\twocolumn[
\icmltitle{Supplement to: Inducing Domain Specific Languages for Bayesian Program Learning}

% It is OKAY to include author information, even for blind
% submissions: the style file will automatically remove it for you
% unless you've provided the [accepted] option to the icml2018
% package.

% List of affiliations: The first argument should be a (short)
% identifier you will use later to specify author affiliations
% Academic affiliations should list Department, University, City, Region, Country
% Industry affiliations should list Company, City, Region, Country

% You can specify symbols, otherwise they are numbered in order.
% Ideally, you should not use this facility. Affiliations will be numbered
% in order of appearance and this is the preferred way.
\icmlsetsymbol{equal}{*}

\begin{icmlauthorlist}
\icmlauthor{Aeiau Zzzz}{equal,to}
\icmlauthor{Bauiu C.~Yyyy}{equal,to,goo}
\icmlauthor{Cieua Vvvvv}{goo}
\icmlauthor{Iaesut Saoeu}{ed}
\icmlauthor{Fiuea Rrrr}{to}
\icmlauthor{Tateu H.~Yasehe}{ed,to,goo}
\icmlauthor{Aaoeu Iasoh}{goo}
\icmlauthor{Buiui Eueu}{ed}
\icmlauthor{Aeuia Zzzz}{ed}
\icmlauthor{Bieea C.~Yyyy}{to,goo}
\icmlauthor{Teoau Xxxx}{ed}
\icmlauthor{Eee Pppp}{ed}
\end{icmlauthorlist}

\icmlaffiliation{to}{Department of Computation, University of Torontoland, Torontoland, Canada}
\icmlaffiliation{goo}{Googol ShallowMind, New London, Michigan, USA}
\icmlaffiliation{ed}{School of Computation, University of Edenborrow, Edenborrow, United Kingdom}

\icmlcorrespondingauthor{Cieua Vvvvv}{c.vvvvv@googol.com}
\icmlcorrespondingauthor{Eee Pppp}{ep@eden.co.uk}

% You may provide any keywords that you
% find helpful for describing your paper; these are used to populate
% the "keywords" metadata in the PDF but will not be shown in the document
\icmlkeywords{Machine Learning, ICML}

\vskip 0.3in
]

% this must go after the closing bracket ] following \twocolumn[ ...

% This command actually creates the footnote in the first column
% listing the affiliations and the copyright notice.
% The command takes one argument, which is text to display at the start of the footnote.
% The \icmlEqualContribution command is standard text for equal contribution.
% Remove it (just {}) if you do not need this facility.

%\printAffiliationsAndNotice{}  % leave blank if no need to mention equal contribution
\printAffiliationsAndNotice{\icmlEqualContribution} % otherwise use the standard text.

\section{Program Representation}\label{programrepresentation}
We choose to represent programs using $\lambda$-calculus~\cite{pierce}.
A $\lambda$-calculus expression is either:
\\\noindent A \emph{primitive}, like the number 5 or
  the function \texttt{sum}.
\\\noindent A \emph{variable}, like $x$, $y$, $z$
\\\noindent A $\lambda$\emph{-abstraction}, which creates a new function. $\lambda$-abstractions have a variable and a body. The body is a $\lambda$-calculus expression. Abstractions are written as $\lambda \text{var}. \text{body}$.
\\\noindent An \emph{application} of a function to an argument. Both the function and the argument are $\lambda$-calculus expressions. The application of the function $f$ to the argument $x$ is written as $f\; x$.

For example, the function which squares the logarithm of a number is
$\lambda x.\text{\texttt{square}} (\text{\texttt{log} } x)$, and the identity function $f(x) = x$ is $\lambda x.x$. The
$\lambda$-calculus serves as a spartan but expressive Turing complete
program representation, and distills the essential features of functional languages like Lisp.

However, many $\lambda$-calculus expressions correspond to ill-typed programs, such as the program that takes the logarithm of the Boolean \texttt{true} (i.e., \texttt{log true}) or which applies the number five to the identity function
(i.e., $5 \; (\lambda x.x)$).
We use a well-established typing system for $\lambda$-calculus called \emph{Hindley-Milner typing}~\cite{pierce}, which is used in programming languages like OCaml.
The purpose of the typing system is to ensure that our programs never call a function with a type it is not expecting (like trying to take the logarithm of \texttt{true}).
Hindley-Milner has two important features:
Feature 1: It supports \emph{parametric polymorphism}: meaning that types can have variables in them, called \emph{type variables}. Lowercase Greek letters are conventionally used for  type variables.
For example, the type of the identity function is $\alpha\to\alpha$, meaning it takes something of type $\alpha$ and return something of type $\alpha$. A function that returns the first element of a list has the type $\texttt{list}(\alpha)\to\alpha$. Type variables are not the same has variables introduced by $\lambda$-abstractions.
Feature 2: Remarkably, there is a  simple algorithm for automatically inferring the polymorphic Hindley-Milner type of a $\lambda$-calculus expression~\cite{damas1982principal}.
A detailed exposition of Hindley-Milner is beyond the scope of this work.

%% \begin{figure}
%%   \begin{align}
%%     \text{primitive types} & 
%%     \end{align}
%%   \end{figure}

\section{Neural recognition model architecture}

\section{Estimating $\theta$}



We write $c(e,p)$ to mean the number of times that primitive $e$ was used in program $p$;  $R(p)$ to mean the sequence of types input to sample in Alg.\ref{programGenerativeModel}. Jensen's inequality gives an intuitive lower bound on the likelihood of a program $p$:
\begin{align*}
  \log \probability[p|\theta]&\stackrel{+}{ = }\sum_{e\in \mathcal{D}} c(e,p)\log \theta_e - \sum_{\tau\in R(p)} \log \sum_{\substack{e:\tau'\in \mathcal{D}\\\text{unify}(\tau,\tau')}}\theta_e\\
  &\stackrel{+}{\geq }\sum_{e\in \mathcal{D}} c(e,p)\log \theta_e - c(p)\log \sum_{\tau\in R(p)} \sum_{\substack{e:\tau'\in \mathcal{D}\\\text{unify}(\tau,\tau')}}\theta_e\\
  & = \sum_{e\in \mathcal{D}} c(e,p)\log \theta_e - c(p)\log \sum_{e\in \mathcal{D}} r(e,p)\theta_e
\end{align}
where $c(p) = \sum_{e\in \mathcal{D}}c(e,p)$ and $r(e:\tau',p) = \sum_{\tau\in R(p)} \indicator[\text{canUnify}(\tau,\tau')]$.

Differentiate with respect to $\theta_e$ and set to zero 
\begin{align}
  \frac{c(x)}{\theta^{(x)}} &= N\frac{a(x)}{\sum_y a(y)\theta_y}
\end{align}
This equality holds if $\theta^{(x)} = c(x)/a(x)$:
\begin{align}
  \frac{c(x)}{\theta_x} &= a(x).\\
N\frac{a(x)}{\sum_y a(y)\theta_y}& = N\frac{a(x)}{\sum_y c(y)}   = N\frac{a(x)}{N} = a(x).
\end{align}
If this equality holds then $\theta_x \propto c(x)/a(x)$:
\begin{align}
  \theta_x = \frac{c(x)}{a(x)}\times \underbrace{\frac{\sum_y a(y)\theta_y}{N}}_{\text{Independent of $x$}}.
\end{align}

Now what we are actually after is the parameters that maximize the joint log probability of the data+parameters, which I will write $J$:
\begin{align}
  J& = L + \log \text{D}(\theta|\alpha)\\
  &\stackrel{+}{\geq } \sum_x c(x)\log \theta_x - N \log \sum_x a(x)\theta_x  + \sum_x(\alpha_x - 1)\log \theta_x\\
  & = \sum_x (c(x) + \alpha_x - 1)\log \theta_x -  N \log \sum_x a(x)\theta_x
\end{align}
So you add the pseudocounts to the \emph{counts} ($c(x)$), but not to the \emph{possible counts} ($a(x)$).


\bibliography{main}
\bibliographystyle{icml2018}

\end{document}


% This document was modified from the file originally made available by
% Pat Langley and Andrea Danyluk for ICML-2K. This version was created
% by Iain Murray in 2018. It was modified from a version from Dan Roy in
% 2017, which was based on a version from Lise Getoor and Tobias
% Scheffer, which was slightly modified from the 2010 version by
% Thorsten Joachims & Johannes Fuernkranz, slightly modified from the
% 2009 version by Kiri Wagstaff and Sam Roweis's 2008 version, which is
% slightly modified from Prasad Tadepalli's 2007 version which is a
% lightly changed version of the previous year's version by Andrew
% Moore, which was in turn edited from those of Kristian Kersting and
% Codrina Lauth. Alex Smola contributed to the algorithmic style files.
