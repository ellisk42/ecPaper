\documentclass{article}

\usepackage{nips_2018_author_response}

\usepackage[utf8]{inputenc} % allow utf-8 input
\usepackage[T1]{fontenc}    % use 8-bit T1 fonts
\usepackage{hyperref}       % hyperlinks
\usepackage{url}            % simple URL typesetting
\usepackage{booktabs}       % professional-quality tables
\usepackage{amsfonts}       % blackboard math symbols
\usepackage{nicefrac}       % compact symbols for 1/2, etc.
\usepackage{microtype}      % microtypography

\begin{document}

Something something thank you guys for your rich on point comments that are so
very useful for making science what it is.

\subsubsection*{Choice of original \texttt{DSL}}

As several reviewers pointed out the original \texttt{DSL} influences the efficiency and
accuracy of the algorithm. Reviewer 1 points out ``What happens, for example, if
length is removed [from the original list \texttt{DSL}]?''. To explore that question we
took the original \texttt{Lisp} specification from [citation needed] as well as
tasks from \texttt{Lisp} textbooks to see whether higher level functions would
be discovered. While this example was computationally expensive, it successfully
rediscovered common abstractions from the functional programming world,
including but not limited to \texttt{fold}, \texttt{unfold}, \texttt{map},
\texttt{length}, \texttt{zip}, \texttt{range}, etc.

That being said, the intended goal of the algorithm is not to rediscover
well-known useful abstractions but rather to adapt such abstractions to the
tasks at hands, through more abstractions if need be but not with that as a goal
\textit{per se}.

\paragraph*{Example of evolution for a task through \texttt{DSL}s}

Find some logs and give an example I guess? I have a few available for geometry
but hey :/

\subsubsection*{Relationship to Schmid and Kitzelmann}

We thank reviewer three for suggesting additions to the related work. Our
revision will prominently highlight the work of Kitzelmann, Schmid and Katayama.

The \textsc{Igor} systems introduced by Kitzelmann and Schmid is a sophisticated
algorithm for analytical inductive programming that handles a certain classe of
programs, much like {Sketch} and {FlashFill} are sophisticated
program synthesiser for certain classes of programs.

We see these line of work as orthogonal and complementary to that of SCC\@. Our
goal is to take a black-box synthesiser and learn how to solve a class of
problems. One could in principle replace our enumerative search based solver
with \textsc{Igor}, Sketch or \textsc{MagicHaskeller}. We chose however to
use a more generic and simple enumerative approach for it allows us to apply SCC
to a diverse spread of problems: functional programming exercises, text editing,
graphics programming and symbolic regressions from images.


\subsubsection*{Reproducibility}

Our project --- including source code, data and automation scripts --- is
already available on \url{GitHub} and future revision will include a link to it
once the anonymity of the authors can be unveiled. No technical knowledge will
be required to reproduced the results.

\end{document}
